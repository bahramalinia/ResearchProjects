\section{Related Work}
\label{sec:rel}
There is an extensive work in the broad topic of EV demand management and scheduling~\cite{mukherjee2015review} such as as optimal operations with EV coordination and congestion management~\cite{rigas2013congestion,hu2013coordinated}, EV scheduling with incorporation of renewable energy, and energy storage systems~\cite{de2017impact,shafie2018innovative}, and pricing and bidding~\cite{huang2014distribution}. 
We focus on the literature related to peak-constrained EV charging scheduling.
\subsubsection{Peak-Constrained EV Charging Scheduling}
\label{sec:rel:peakconstrained}			
There is an extensive literature on EV charging scheduling problem focusing on single station~\cite{Tang,Wen}, while the local and global peak constraints are omitted or only the local peak is considered. As we discuss in Section~\ref{sec:problem}, the global optimal solution cannot be obtained by separately solving the single station problems. Hence, those solutions cannot be directly applied to the multiple station scenario with global peak constraints. 

%A scenario that local and global peak constraints exist is the case that scheduling is required for a charging network for multiple stations. 
Studies in \cite{He,malhotra2017distributed,Moradijoz,DWang,Zeng,Shaaban} tackled charging scheduling problem in multiple stations. The authors in~\cite{He} studied a global cost minimization EV charging scheduling problem, without taking into account the the maximum peak demand that the system can tolerate.  
%We solve the issue by constraining local and global peaks. However, to meet the peak constraints, it may not be feasible to respond to all charging demands. Consequently, only a subset of EVs can be charged~\cite{Xiang}, which is captured in our problem formulation. 
\cite{DWang,Zeng,Shaaban} considered an \textit{offline} multi-microgrid system with global peak constraint where each microgrid has a station and the goal is to manage electricity exchange between microgrids to minimize the operating costs. The authors assume that required information about EVs is available by forecast which may not represent a real scenario. 

%A similar assumption is made in \cite{Zeng,Shaaban} where the objective is to maximize total utility of EVs and aggregators in a distribution network.
%\rev{To be removed: In our solution, we applied a priority based selection process (based on the EVs' valuation) to respond to the demands to make sure that total peak respects the constraints. The selection process can be made based on different criteria such as the value of each charging request \cite{Lee, Xiang} or the priority.}
%To avoid big billing cost in peak hours, \cite{Moradijoz} proposed a solution based on genetic algorithm to find optimal capacity and location of parking lots for serving demands in peak hours with the goal of maximizing total benefit of all stations. Although authors studied the problem under multiple station setting, their solution is not applicable to our setting when the stations are already set up. 
The authors in~\cite{malhotra2017distributed} use a similar model as in this paper, where both local and global peak constraints are considered. However, the authors solve the single-slot problem, which fails to provide a general solution taking into account EVs' arrival and departure times as considered in our study.
Finally, as an alternative approach to control the peak~\cite{Zhang2015eEnergy,lee2019learning}, some studies directly target minimizing the peak~\cite{Zhao}. 
%In~\cite{Zhao}, an online algorithm is developed for EV charging to minimize the peak and \cite{Karfopoulos} proposed a valley filling method by leveraging V2G in peak hours. 
Although the peak is minimized in above works, it cannot guarantee that the minimized peak is tolerable by the ACN.


\subsubsection{Scheduling Under Demand Uncertainty}
A main challenge in EV scheduling problem is to cope with demand uncertainty. Many studies including \cite{Shroff2014,Tang,WTang,Chen,Xiang,Zhao,Robu,Hajiesmaili2016Online,hajiesmaili2017crowd} addressed online scheduling problem with different objectives, including social welfare, station revenue maximization, and maximum utilization of renewable sources. 

%\cite{Shroff2014,Xiang} studied the problem of maximizing social welfare considering the benefit for both users and service provider. \cite{Tang} and \cite{WTang} developed algorithms to minimize the charging price for the station, where the proposed algorithm in \cite{Tang} is $2.39$-competitive. 
%An online algorithm developed in \cite{Chen} to optimize overall operating profit of the service provider. 
%In \cite{Robu}, an online $2$-competitive algorithm is proposed for a single station, assuming that EVs are capable of being discharged in a negligible amount of time. The assumption, however, is not realistic for EVs.

Our problem in this paper is unique from above works in many respects. First, we study the problem in an ACN where several stations exist, while none of the above studies solve the problem under this setting. Second, the previous algorithms do not work for both integral and fractional charging models. In more relevant theoretical problems, \cite{WTang,Chen} put no limit on the charging rate of EVs which makes their solution impractical in real scenarios. Also, \cite{WTang,Tang, Shroff2014,Chen,Zhao} do not consider the peak limit of the station. 
%\revv{Note that some studies~\cite{jin2017optimal} tackle the scenarios of satisfying the large demands (possibly EV demands) by participating in electricity market. This study assumes that the electricity price is given and fixed, and leaves the case of real-time pricing as the future direction. In addition, there are studies that incorporate prediction in online EV charging scheduling~\cite{wang2017predictive,chen2014distributional}. The prediction-based approaches achieve satisfactory performance for the scenarios that follow prediction. Deviation from prediction models, however, degrades their performance. Our approach, on the other hand, has no assumptions on modeling/prediction, and in this way, is robust against any uncertainty in the instances to the problem.} 
Finally, in~\cite{Alinia2018ITS}, we considered a simplified EV charging scheduling in fractional model without global peak constraints and devised heuristic algorithms (without competitive and approximation analysis) with on-arrival commitment for EVs to notify the amount that they can receive by their departure.
%In this paper, we consider these limitations and develop online and offline algorithms for fractional and integral revenue models in an ACN and provide theoretical bounds on their performance.



%It is important to note that there is a considerable similarity between EV scheduling problem and classical job scheduling problem \cite{}. However, most algorithm developed for job scheduling problem  do not consider a limit on processing rate of the jobs. 

%\cite{Shroff2014} and \cite{Stein2012} study social welfare maximization problem while the profit of both EV owners and station is considered.  social welfare of the EV owners \cite{Stein2012}, charging cost \cite{Tang,WTang}, and total value of fully charged EVs \cite{Azar}). In addition, \cite{WTang, Tang,Shroff2014,Zhao} do not take into account the peak limitation of station. }

\subsubsection{Worst-case Analysis in Similar Scheduling Problems}
\label{sec:rel:comp}
	Similar underlying scheduling problems with slightly different settings have been studied in the literature in both offline and online settings. In offline setting, the problem is more interesting under integral revenue model where the problem becomes a combinatorial optimization problem and approximation algorithms have been used to find effective solutions. 
The performance of an approximation algorithm is determined by its \emph{approximation ratio} for offline 
algorithms. On the other hand, competitive online algorithms are used in the online setting and \emph{competitive ratio} is the performance metric which compares the algorithm's result to the offline optimal solution. 

\textit{Offline integral model}: Under integral revenue model, \cite{Jain} proposes an offline algorithm for scheduling of batch jobs in cloud computing which is similar to EV scheduling problem. It is assumed that all jobs are available to be processed at time 0. The authors propose a $\frac{C}{C-k}.\frac{s}{s-1}$-approximation algorithm where $C$ is the cloud capacity and $s$ is the ``slackness parameter'' (see section \ref{sec:acn} for the definition). 
Similarly, \cite{yao2016real} tackles the offline integral problem and proposes a convex relaxation method to find a near-optimal solution. No theoretical bound is provided for the algorithm and the performance is examined by simulation results. In this paper, we propose \ics for the EV scheduling problem with slightly different settings in the constraint sets, and tackle the problem in online setting and provide approximation analysis under several scenarios. 

%\textit{Offline fractional model}: In case of fractional revenue model (for offline case), the scheduling problem is more straightforward to tackle since the underlying problem is linear. In this paper, we propose \fcs as an optimal solution for offline fractional case with low time complexity.

\textit{Online fractional model}: In our recent study \cite{alinia2018competitive}, 
we proposed two online algorithms referred to as WFAIR and WRAND for EV scheduling problem. The proposed algorithms provide a competitive ratio of $2-1/U$ where $U$ is "scarcity level". However, the result only holds when all EVs have the same maximum charging rate. Also, \cite{alinia2018competitive} does not study the integral revenue model. In this paper, we propose \focs as an online algorithm. Moreover, the current study investigates both fractional and integral revenue model while considering total peak constraint in the formulation and addressing multiple charging station scenario. 
In another work~\cite{firstfit}, two simple and natural online algorithms called \textsc{FirstFit} and \textsc{EndFit} were developed and are proved to be 2-competitive. \focs is an extension of these algorithms by taking into account the maximum charging rate of the EVs and multiple charging station scenario (see ``Remarks'' at Section IV). As recent study, \cite{zheng2016online} provides online algorithms for EV scheduling problem, however, they do not take into account global peak constraint in the underlying problem. 

\textit{Online integral model}: Another direction is to tackle online scenario under integral revenue model, where the scheduling problem becomes fundamentally more challenging. We note that the integral scheduling problem is strongly NP-Hard even in the offline case \cite{de2018complexity}. Due to combined online and combinatorial challenges, there are very limited studies in this category. \cite{lucier2013efficient} provides an online algorithm without considering the maximum processing rate of the jobs where its competitive ratio can be arbitrary bad depending on ``slackness'' parameter. 
Our work extends this study for multiple station scenario and considering maximum charging rate for the EVs.

