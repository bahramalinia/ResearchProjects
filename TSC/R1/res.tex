\documentclass[11pt]{article}

\usepackage[paper=a4paper,dvips,top=4cm,left=2.5cm,right=2.5cm,
foot=2cm,bottom=4cm]{geometry}
%\usepackage{float}
\usepackage[linesnumbered,ruled,vlined]{algorithm2e}
%\usepackage{algorithm}
%\usepackage{algorithmic}
\usepackage{amssymb,amsmath}
\usepackage{caption}
%\usepackage{subcaption}
\usepackage{comment}
\usepackage{color,soul}
%\usepackage{ragged2e} 
%\usepackage[usenames,dvipsnames]{xcolor}
%\usepackage{subfigure}
%\usepackage{flushend}
% *** CITATION PACKAGES ***
%
\usepackage{cite}
\usepackage[pdftex]{graphicx}
\graphicspath{{figRes/}{../}}

\usepackage[caption=false,font=footnotesize]{subfig}
%
% correct bad hyphenation here
\hyphenation{op-tical net-works semi-conduc-tor}
\setlength{\intextsep}{-1ex} % remove extra space above and below in-line float

\newcommand{\sa}{\textsc{SmartAllocate}}
\newcommand{\rc}{\textsc{ReConsider}}
\newcommand{\bc}{\textsc{BetaCover}}
\newcommand{\ics}{\textsc{iCS}\xspace}
\newcommand{\fcs}{\textsc{fCS}\xspace}
\newcommand{\MCSP}{\textsf{SPAN}\xspace}
\newcommand{\dMCSP}{\textsf{dSPAN}\xspace}
\newcommand{\focs}{\textsc{foCS}\xspace}
\newcommand{\iocs}{\textsc{ioCS}\xspace}
\newcommand{\finalalg}{\textsc{ioCS+}\xspace}
\newcommand{\opt}{\textsc{OPT}\xspace}
\newcommand{\alg}{\textsc{ALG}\xspace}
\newcommand{\rtl}{\textsc{GreedyRTL}\xspace}
\newcommand{\iolp}{\textsc{ioLP}\xspace}
\newcommand{\folp}{\textsc{foLP}\xspace}

\ifodd 1
\newcommand{\rev}[1]{{\color{black}#1}}%revise of the text
\newcommand{\com}[1]{\textbf{\color{red}(Bahram says: #1)}}%comment of the text
\newcommand{\comm}[1]{\textbf{\color{red}(Mohammad says: #1)}}%comment of the text
\else
\newcommand{\rev}[1]{#1}
\newcommand{\com}[1]{}
\fi

\usepackage[colorinlistoftodos, textwidth=4cm, shadow]{todonotes}
\newcommand{\bahram}[1]{\todo[inline,color=orange!40]{{\it Bahram:~}#1}}
\newcommand{\enrique}[1]{\todo[inline,color=blue!15]{#1}}
\begin{document}


\title{Response Letter for manuscript TSG-01885-2017 ``Online EV Scheduling Algorithms for Adaptive Charging Networks with Global Peak Constraints'' \\
	\vspace{4mm} \large
	by  Bahram~Alinia, Mohammad~H.~Hajiesmaili, Zachary J. Lee, Noel Crespi, and Enrique Mallada
}

\maketitle

\textbf{Dear Editor,}

We are very grateful for handling our paper and the time and effort that you and your team put into the process. We have thoroughly revised our manuscript based on your and the referees' constructive comments. This document provides a summary of changes made in the revised manuscript and detailed responses to the comments of the reviewers. The major changes in the revised manuscript are summarized below:

\begin{enumerate}
\item \textit{...}. 


\end{enumerate}


In what follows, we mention first the comments (as appeared in the decision letter, highlighted in {\color{blue} blue} in this letter) followed by a description on how we addressed those comments in the paper. Once again, thank you and your team for providing very constructive comments to help in improving this paper.



%%%%%%%%%%%%%%%%%%%%%%%%%%%%%%%%%%%%%%%%%%%%%%%%%%%%%%%%%%%%%%%%%%%%%%%%%%%%%%
\newpage

%\renewcommand{\thesection}{\arabic{count})}
%\newcounter{count}
%\stepcounter{count}
%\addtocounter{section}{1}
%\setcounter{section}{1}
{\Large\textbf{Editor's Comments:}}
\vspace{3mm}

{\color{blue}The reviewers have identified several issues with the paper that lie mainly with a lack of justification for some of the assumptions made which restrict the value of the theoretical results produced. Also the authors need to clearly delineate this work from their previous work. A major revision of this paper is recommended.}

\vspace{5mm}
\noindent\textbf{Response:}



%%%%%%%%%%%%%%%%%%%%%%%%%%%%%%%%%%%%%%%%%%%%%%%%%%%%%%%%%%%%%%%%%%%%%%%%%%%%%%
%%%%%%%%%%%%%%%%%%%%%%%%%%%%%%%%%%%%%%%%%%%%%%%%%%%%%%%%%%%%%%%%%%%%%%%%%%%%%%
%%%%%%%%%%%%%%%%%%%%%%%%%%%%%%%%%%%%%%%%%%%%%%%%%%%%%%%%%%%%%%%%%%%%%%%%%%%%%%
%%%%%%%%%%%%%%%%%%%%%%%%%%%%%%%%%%%%%%%%%%%%%%%%%%%%%%%%%%%%%%%%%%%%%%%%%%%%%%
%%%%%%%%%%%%%%%%%%%%%%%%%%%%%%%%%%%%%%%%%%%%%%%%%%%%%%%%%%%%%%%%%%%%%%%%%%%%%%
%%%%%%%%%%%%%%%%%%%%%%%%%%%%%%%%%%%%%%%%%%%%%%%%%%%%%%%%%%%%%%%%%%%%%%%%%%%%%%
%%%%%%%%%%%%%%%%%%%%%%%%%%%%%%%%%%%%%%%%%%%%%%%%%%%%%%%%%%%%%%%%%%%%%%%%%%%%%%
%%%%%%%%%%%%%%%%%%%%%%%%%%%%%%%%%%%%%%%%%%%%%%%%%%%%%%%%%%%%%%%%%%%%%%%%%%%%%%
%%%%%%%%%%%%%%%%%%%%%%%%%%%%%%%%%%%%%%%%%%%%%%%%%%%%%%%%%%%%%%%%%%%%%%%%%%%%%%
%%%%%%%%%%%%%%%%%%%%%%%%%%%%%%%%%%%%%%%%%%%%%%%%%%%%%%%%%%%%%%%%%%%%%%%%%%%%%%
\newpage
\section{Reviewer $\# 1$}

{\color{blue} The authors fail to explain the difference of the first part on the fractional model from their earlier work: several 2-competitive scheduling methods are listed by Alinia et al. (2018) and references, including an algorithm with a better competitive ratio and/or runtime. I think this should not have been included in this paper. }
\vspace{3mm}

$\vartriangleright$ \noindent\textbf{Response:} 
Our earlier work presents algorithms for only fractional revenue model. Current manuscript, which can be considered as journal extension of the previous work, studies both fractional and integral models. Although the presented algorithms in the conference version provide a competitive ratio of $2-1/U$ (with $U$ as "scarcity level"), the proof holds when all EVs have the same maximum charging rate. In the current study, we consider hetregenousity in this parameter and prove the online algorithm is 2-competitive.

To clarify this, we added an explanation in Section ...   

\vspace{3mm}
{\color{blue} Furthermore, why is there no inclusion of electricity prices and/or value for demand response, which is a particular powerful use case for EV charging scheduling? More generally speaking, please add a proper motivation of the two models: when would we use which model in practice? }
\vspace{3mm}

{\color{red}: @Mohammad: This is not clear for me in most parts} 

$\vartriangleright$ \noindent\textbf{Response:} 
The variable $v_i$ defined in the paper is valuation of EV $i$ when it receives its demand $D_i$. In simple and default case $v_i$ can represent the price imposed by charging station that the owner of EV $i$ should pay to receive its demand, $D_i$ (as it is in the paper). In this case, the charging station is allowed to use any pricing policy as it does not affect our algorithms. With this interpretation, our algorithms solve revenue maximization problem for utility provider. $v_i$ can also represent the valuation of demand from the user's point of view indicating that how much the user will be happy if it receives its submitted demand. With this interpretation, our algorithms try to maximize users' satisfaction or users' social welfare. Therefore, our proposed algorithms are general and work for any valuation setting that is proposed by either EVs or charging stations.
	
\vspace{3mm}
{\color{blue} The approach for the integral model seems novel to me, and it is nice that a theoretical analysis is included. However, the theoretical results are disappointing, which reduces their value a bit: they only hold under severe restrictions: Thm.4 only holds "when EVs have same arrival time" ... which is quite unrealistic. Thm.5 only holds for m=1, which is the single constraint often found in literature. (Of which the authors write: [our] "problem is different from EV charging scheduling with capacity constraint in single station scenarios [2], [4], [7]–[12]".). Moreover, the ratios are pretty weak (for Thm.4: number of charging stations +1 times the optimal, for Thm.5: number of different arrival slots + 1 times the optimal). }
\vspace{3mm}

$\vartriangleright$ \noindent\textbf{Response:} 
...

\vspace{3mm}
{\color{blue} Actual useful proof of the quality of the proposed algorithm can be found in the experiments. I think the paper could be interesting if this section received significantly more attention, for example
 - showing the effect of problem parameters (range in arrival times, departure times, values, demand, number of vehicles, etc.) on the performance of the  algorithms (not only value but also computation time)
 - comparing to other methods from the literature, e.g. Yao et al. (2017) use a charging priority (think of this as your value) and adding the extra CS constraints is pretty straightforward for such MILP models. Moreover, Yao et al. (2017) also include prices as well as demand response, but also these can of course just be ignored.
 }
\vspace{3mm}

$\vartriangleright$ \noindent\textbf{Response:} 
{\color{red}@Mohammad: We should try to distinguish our work from Yao et al. but it seems difficult.}

\vspace{3mm}
{\color{blue} The implementation of the algorithms and the benchmark problems would be helpful if submitted as supplementary files. }
\vspace{3mm}

$\vartriangleright$ \noindent\textbf{Response:} 
We listed pseudo-codes for all proposed algorithms and an interested reader can easily convert them to real implementation code. We believe that detailed implementation of the algorithms has no added value as the goal here is to provide algorithms and analysis and not focusing on programming challenges.
  
\vspace{3mm}
{\color{blue} On page 2 and page 7 you write that "the underlying problem is NP-hard due to 0/1 selection" and you mention knapsack, but in fact knapsack is one of the easier NP-hard problems. It seems that your integral model models a strongly NP-hard problem, see De Weerdt et al. (2018). }
\vspace{3mm}

$\vartriangleright$ \noindent\textbf{Response:} 
Thanks for the point. We agree that our problem is strongly NP-Hard and we corrected the related phrases in the manuscript.

\vspace{3mm}
{\color{blue} On page 4, l19, the constraint (1d) seems to force equal charging speed from the arrival time to the departure time. I don't think this constraint is met by the greedy algorithms, is it? Then this is not your "Scheduling Problem for Adaptive charging Network (SPAN) under fractional revenue model"? }
\vspace{3mm}

$\vartriangleright$ \noindent\textbf{Response:} 
...

\vspace{3mm}
{\color{blue} p4: The discussion of the offline algorithm could benefit from an example why straightforwardly allocating the most valuable ($v_i/D_i$) is not optimal (because of charging speed limits), and then introduce the greedy re-allocation (based on flexibility). }
\vspace{3mm}

$\vartriangleright$ \noindent\textbf{Response:} 
...

\vspace{3mm}
{\color{blue} In the experimental evaluation the results for both models are put together in the same figure, which gets us into the odd situation that some results are better than optimal.
 }
\vspace{3mm}

$\vartriangleright$ \noindent\textbf{Response:} 
...

\vspace{3mm}
{\color{blue} typos:
p3: "We formulate Scheduling"

p3:l33: "ptotal, which we refer to it as the global peak, hereafter" => I think you mean to say that this is the capacity limit, not the global peak.

p4: "a definitions"

p7: "If there is not enough resources"

p7: "In primal-dual algorithm"

note [page 8]: Other approximations?

p8: "under integral mode"

p8: "for fractional mode"

p8: "We propose the IOCS"

p8: "over set of"

p8: "with optima"

p8: "in accordance to NHTS survey"

p8: "of arriving an EV in the peak hours"

p9: "in integral revenue

p9: "The proposed algorithms compared to"

p9: "IO LP.Recall"
 }
\vspace{3mm}

$\vartriangleright$ \noindent\textbf{Response:} 
...


\begin{itemize}
\item B. Alinia, M. S. Talebi, M. H. Hajiesmaili, A. Yekkehkhany and N. Crespi, Competitive Online Scheduling Algorithms with Applications in Deadline-Constrained EV Charging, IEEE/ACM 26th International Symposium on Quality of Service (IWQoS), 2018
doi: 10.1109/IWQoS.2018.8624184

\item de Weerdt, M. M., Albert, M., Conitzer, V., \& Linden, K. V. D. (2018). Complexity of scheduling charging in the smart grid. In Proceedings of the Twenty-Seventh International Joint Conference on Artificial Intelligence (pp. 4736-4742).

\item Yao, L., Lim, W. H., \& Tsai, T. S. (2017). A real-time charging scheme for demand response in electric vehicle parking station. IEEE Transactions on Smart Grid, 8(1), 52-62.
\end{itemize}

\newpage
\section{Reviewer $\# 2$}

\vspace{3mm}
{\color{blue} 1.      The authors need to examine the literature more thoroughly. Although the proposed methods have a certain degree of novelty, more related works exist. For example, the authors could consider some of the following papers, while others may also exist.

\begin{itemize}
\item a: Huang, S., Wu, Q., Oren, S. S., Li, R., \& Liu, Z. (2015). Distribution locational marginal pricing through quadratic programming for congestion management in distribution networks. IEEE Transactions on Power Systems, 30(4), 2170-2178.

\item b: Hu, J., You, S., Lind, M., \& Østergaard, J. (2014). Coordinated charging of electric vehicles for congestion prevention in the distribution grid. IEEE Transactions on Smart Grid, 5(2), 703-711.

\item c: Rigas, E. S., Ramchurn, S. D., Bassiliades, N., \& Koutitas, G. (2013, October). Congestion management for urban EV charging systems. In 2013 IEEE International Conference on Smart Grid Communications (SmartGridComm) (pp. 121-126). IEEE.

\end{itemize}
}
\vspace{3mm}

$\vartriangleright$ \noindent\textbf{Response:} 
...

\vspace{3mm}
{\color{blue} The authors claim that if prediction or modelling techniques were used, the performance of the proposed algorithms would deteriorate. However, they do not present any results to support their claim.  }
\vspace{3mm}

$\vartriangleright$ \noindent\textbf{Response:} 
...

\vspace{3mm}
{\color{blue} An interesting extension would be to let users define a set of CSs to charge and let the scheduling algorithms take the final decision of the CS to charge. In this way, the flexibility of the algorithms would increase and the load across the CSs could be better managed. However, the downside would be that the complexity of the algorithms would increase. }
\vspace{3mm}

$\vartriangleright$ \noindent\textbf{Response:} 
...

\vspace{3mm}
{\color{blue} Is truth telling regarding the EVs’ valuations guaranteed? Is it possible for the EVs to cooperate and report lower valuations? It is not clear in the paper. If not, this could come in contrast with the revenue maximization objective function. Moreover, this would cause serious implications in the online algorithms which use the valuations to select the EVs to charge.     }
\vspace{3mm}

$\vartriangleright$ \noindent\textbf{Response:} 
...

\vspace{3mm}
{\color{blue} The length of each time point is set to 1 hour. This is too long, and the accuracy of the algorithms is reduced. Moreover, the online algorithm is not really online, as it schedules EVs every hour. }
\vspace{3mm}

$\vartriangleright$ \noindent\textbf{Response:} 
...

\vspace{3mm}
{\color{blue} Some results on execution times and scalability of the proposed algorithms must be added.
 }
\vspace{3mm}

$\vartriangleright$ \noindent\textbf{Response:} 
...

%\bibliography{ref-response}{}
%\vspace{-3mm}
%\bibliographystyle{ieeetr}

\end{document}