\documentclass[11pt]{article}

\usepackage[paper=a4paper,dvips,top=4cm,left=2.5cm,right=2.5cm,
foot=2cm,bottom=4cm]{geometry}
%\usepackage{float}
\usepackage[linesnumbered,ruled,vlined]{algorithm2e}
%\usepackage{algorithm}
%\usepackage{algorithmic}
\usepackage{amssymb,amsmath}
\usepackage{caption}
\usepackage{subcaption}
\usepackage{comment}
\usepackage{color,soul}
\usepackage{ragged2e} 
\usepackage[usenames,dvipsnames]{xcolor}
%\usepackage{subfigure}
%\usepackage{flushend}

% *** CITATION PACKAGES ***
%
\usepackage{cite}
%\usepackage[pdftex]{graphicx}


\usepackage[caption=false,font=footnotesize]{subfig}
%
% correct bad hyphenation here
\hyphenation{op-tical net-works semi-conduc-tor}
\setlength{\intextsep}{-1ex} % remove extra space above and below in-line float
\newcommand{\SCS}{\textsc{iCS}\xspace}
\newcommand{\SCSP}{\textsc{fCS}\xspace}
\newcommand{\MCSP}{\textsf{MCSP}\xspace}
\newcommand{\dMCSP}{\textsf{dMCSP}\xspace}
\newcommand{\focs}{\textsc{foCS}\xspace}
\newcommand{\iocs}{\textsc{ioCS}\xspace}
\newcommand{\fcs}{\textsc{fCS}\xspace}
\newcommand{\ics}{\textsc{iCS}\xspace}

\ifodd 1
\newcommand{\com}[1]{{\color{ForestGreen} #1}}%revise of the text
\else
\newcommand{\com}[1]{#1}
\fi

\begin{document}


\title{Response Letter for manuscript VT-2016-01986: ``Microgrid Revenue Maximization by Charging Scheduling of EVs in Multiple Parking Stations'' \\
	\vspace{4mm} \large
	by  Bahram~Alinia, Mohammad~H.~Hajiesmaili, and Noel Crespi
}

\maketitle

We are very grateful to the referees for their thorough reviews of our paper and their constructive comments. Based on their suggestions, we have greatly improved the quality of our paper. This document provides a summary of changes made in the revised
manuscript and detailed responses to the comments of the reviewers. The main changes in the revised manuscript are summarized below:

\begin{enumerate}
\item Based on the critical comments of the reviewers regarding the practicality of the system model and randomness in EVs' demand, we added Section VI ``Online Solution'' to the revised manuscript that provides efficient online and practical algorithms for the studied problems. 

\item The entire simulation section is revised to add the result of the new online algorithms along with the previous offline solutions.

\item The sections "Related Works" and "Introduction" are revised. Specially, we cited new references to position our work and clarify the contributions. 

\end{enumerate}


In what follows, we mention first the comments (as appeared in the decision letter) followed by a description on how we addressed those comments in the paper.

\newpage
\section{Reviewer $\# 1$}
{\color{blue}In this work the authors consider the scheduling of the EV charging in multiple stations which are owned by a single owner. The objective is to minimize the peak value of the aggregated load. This problem is formulated as a mixed integer programming problem. A primal-dual method is developed to approximately solve this problem in polynomial time. Some property of the method is studied theoretically and demonstrated numerically. Overall the paper is clearly written and well organized. Most related references have been appropriately cited. But the current manuscript may be further improved in the following aspects.}

\subsection{Major Comments}
%We sincerely appreciate you for taking the time to review our paper. Following the
%concerns and suggestions, the paper has carefully been revised to address the reviewer's
%comments properly. 

\vspace{5mm}
{\color{blue}
\noindent\textbf{Comment 1.1:}\\
It is not clear how to handle the randomness in the load. A critical assumption in both the problem formulation as well as the theoretical analysis is the perfect knowledge of the arrival time (as well as the departure time) of the electric vehicles. This can hardly be true in practice. The authors need to discuss how the uncertainty in this knowledge may affect the performance of the proposed method. This can at least be done numerically.
}

\vspace{5mm}
\noindent\textbf{Response:}\\
We agree with the reviewer that the assumption of having perfect knowledge of EVs' charging profile is not realistic. To address the issue, in the revised version we added a new section (Section VI on Page $8$) that provides ``online'' algorithms for the problems such that the charging profile of an EV (i.e, its arrival time, deadline, valuation, demand, and maximum charging rate) is not revealed to the charging station before the EV arrives to the station and plugged in. We designed two online algorithms in which we have no assumption on the probabilistic modeling of future arrivals. The new algorithms (\iocs for integral revenue model and \focs for fractional revenue model) are simple yet efficient greedy algorithms that at each time slot only plan for available EVs. The time complexity of online algorithms is $O(n^2T)$ (See Table \ref{tbl:tbl1}) where $n$ is the total number of EVs in all charging stations and $T$ is the number of time slots. Based on the proposed algorithms, the whole simulation section is revised accordingly to include the result for the new algorithms along with the results of previous offline algorithms (i.e., \ics and \fcs). According to the results (See Fig. $1$ in the revised manuscript on Page $10$), \iocs (which is the online version of the \ics algorithm) is $83\%$ and $81\%$ of the \ics and optimal solution of integral revenue model on average, respectively. Moreover, \focs is $80\%$ of its offline optimal solution, \fcs, on average.

	\begin{table}[h]
		\centering
		\vspace{3mm}
		\caption{Proposed algorithms}
		\label{tbl:tbl1}
		\begin{tabular}{ | l | c | c | c |}
			\hline
			\textbf{Algorithm} & Revenue model & Online or Offline & \textbf{Time complexity} \\ \hline\hline    
			\fcs& fractional& offline& $O(n^2T+nT^2)$\\\hline  
			\ics& integral& offline& $O(nT\log T+n^2T)$ \\\hline 
			\focs& fractional& online& $O(n^2T)$\\\hline 
			\iocs& integral&  online& $O(n^2T)$\\\hline
		\end{tabular}
	\end{table}
\vspace{5mm}
%\setcounter{table}{0}
\renewcommand{\thetable}{\Alph{table}}
\vspace{4mm}
{\color{blue}
\noindent\textbf{Comment 1.2:}\\
 As a follow-up with the above comment, it would be a good idea to review some literature that considers the uncertainty in the EV charging problem.}

\vspace{7mm}
\noindent\textbf{Response:}\\
We thank the reviewer for this comment. In light of this comment, we added Subsection II-4 on Page $3$ (Scheduling Under Demand Uncertainty). We added new references to the paper and in this subsection, we reviewed online solutions including [17],[26],[29],[30],[33-37] and clearly positioned our contribution. 
%%%%%%%%%%%%%%%%%%%
\newpage
\section{Reviewer $\# 2$}
{\color{blue}The paper is in good written. The approach has demonstrated well, and the result is clear.}

\subsection{Major Comments}

\vspace{5mm}
{\color{blue}
\noindent\textbf{Comment 2.1}\\
To improve the paper further, stochastic factors are recommended to added into the problem formulation. This may result more complicated problem, but it would be more interesting.}

\vspace{7mm}
\noindent\textbf{Response:}\\
We thank the reviewer for pointing this out. We made all necessary changes to add stochastic factors to our problem model, and proposed \focs and \iocs algorithms that work in online scenario such that the information of the EVs reveals only after their arrival. The new algorithm make decisions only based on the information of available EVs. For more details, please see the response for Comment 1.1. 

\vspace{5mm}
{\color{blue}
\noindent\textbf{Comment 2.2}\\
The paper cited sufficient papers, but reference [35] is not referred in the content.
}

\vspace{7mm}
\noindent\textbf{Response:}\\
Thanks for the subtle comment. The reference is removed in the revised manuscript, and verified the other references. 
%%%%%%%%%%%%
\newpage
\section{Reviewer $\# 3$}
{\color{blue}This paper addresses an interesting and timely topic. However the reviewer has the following comments and concerns.}

\subsection{Major Comments}

\vspace{5mm}
{\color{blue}
\noindent\textbf{Comment 3.1}\\
There is a great amount of existing work on EV charging scheduling in distribution systems (including Microgrids). More important references must be provided in Section I-B and Section II.}

\vspace{7mm}
\noindent\textbf{Response:}\\
We thank the reviewer for his/her comment. To respond to the reviewer's concern, we added more references to Section I-B on Page $1$ (regarding related studies for microgirds) and Section II (regarding online algorithms for EV scheduling problem on Page $3$).

\vspace{5mm}
{\color{blue}
\noindent\textbf{Comment 3.2}\\
Especially, please clearly distinguish this paper with other publications. Otherwise, it is very difficult to identify any new technical contributions of this paper.}

\vspace{7mm}
\noindent\textbf{Response:}\\
Although the EV scheduling problem received substantial attention by researchers during the last years, we believe that the related studies are different in problem formulation and their solution cannot be applied to the unique setting in our work. Some shortcomings one can observe in the previous studies are as follows:

\begin{itemize}

\item \emph{Unlimited charging speed for EVs}: Many studies such as [12],[21],[23-26],[28], %\cite{Malhotra,Jiang,Wen,YCao,TZhang,Akhavan,Lee,WTang} 
assume that EVs can get charged with unlimited charging speed. Notice that considering the charging rate limitation is crucial to make a solution practical for EV scheduling problem. If the charging rate constraint could be eliminated, many of already existed solutions in job scheduling domain such as the well known YDS algorithm 
(``A scheduling model for reduced CPU energy," \emph{in
Proc. of IEEE FOCS}, 1995) for CPU scheduling problem could be directly applied to EV scheduling problem. The charging speed constraint appears in our problem formulation, and our solutions consider it.

\item \emph{Unlimited resource in charging station:} As explained in Section II-2 of the manuscript on Page $3$, the charger devices installed in the parking stations have limitation
on the maximum electricity that they can transfer at each time unit. However, many studies 
[30],[31],[36],[37]
%\cite{Zhao,WTang,Chen,Karfopoulos} 
do not put a constraint on the peak of charging station which cannot represent a real-world problem. The constraint in (1c) in the manuscript represents the resource limitation at each charging station. Also, constraint in (1b) corresponds to total peak limitation in microgrid.

\item \emph{Revenue model:} We consider two charging models namely integral and fractional revenue models (see Section III-A-3 on Page $4$) and develop solutions under both models which is not covered in the previous studies. Each charging model has its advantages. For example, in some scenarios if users get $90\%$ of their demand, they are still satisfied. But this may not be satisfactory in another scenario that a user needs to receive all its demand to reach to its desired destination. Covering both charging models makes our solutions comprehensive and applicable to a wider range of practical scenarios.

\item \emph{Single station solution:}
As elaborated in Section II-1 on Page $2$,  most of previous solutions [8], [12], [18-23] only work under single charging station scenario while the global optimal solution cannot be obtained by separately solving the single station problems as discussed in Section IV.

\end{itemize}


In our paper, we formulate microgrid revenue maximization problem under local and global peak constraints while addressing the aforementioned limitations in the previous studies. Notice that the problem is not studied yet in its current form. We then develop two offline and two online algorithms for both fractional and integral revenue models summarized in Table \ref{tbl:tbl1}. 


Our proposed offline algorithm for fractional revenue model, \fcs, is proved to be optimal.
We also provided  a competitive ratio\footnote{Note that competitive ratio is a well-established metric in theoretical computer science for evaluating the performance of algorithms with no or partial future information. } for the online version of \fcs (i.e., \focs) with the details given in Theorem $5$ on Page $9$. The improvement of the competitive ratio is part of our future studies. Moreover, for \ics algorithm (developed for integral revenue model) and when all EVs have the same arrival time, we proved an approximation ratio of $\alpha = {\Big( 1+\sum_{j=1}^m {\frac{p_j}{p_j-q_j}}.\frac{s}{s-1}\Big)}$, where $p_j$ is local peak constraint in station $j$, $q_j$ is the maximum charging rate of the EVs in station $j$ and $s$ is a slackness parameter.  

In the revised manuscript, we modified the Introduction and Related Work Sections to distinguish our contribution with the existing literature clearly. 

\vspace{5mm}
{\color{blue}
\noindent\textbf{Comment 3.3}\\
The problem formulation is quite standard. The reviewer is not quite sure whether the proposal methods/algorithms are suitable for a realistic scenario in terms of computational cost, accuracy, complexity, and etc.}

\vspace{7mm}
\noindent\textbf{Response:}\\
We revised the paper carefully to clearly highlight efficiency of our proposed solution regarding the important concerns raised in this comment: 
\begin{itemize}
	\item Regarding how algorithms are suitable for a realistic scenario: we agree that in the previous version of our paper there were concerns related to practicality of the proposed methods. Hence, we solved the issue by proposing two new algorithms that work in online scenario in which the decision making is based on the available information and there is no assumptions on demand profile of future coming EVs. In this way, we resolved one of the most important practicality issues of our method. The detailed explanation of the online algorithms is given in Section~VI of the revised paper on Pages~8-9 and the response to Comment 1.1. 
	\item Regarding computational cost and complexity: the time complexities of all proposed algorithms (two online and two offline) are polynomial and are presented in Table \ref{tbl:tbl1} of the response letter and Table II of the revised manuscript on Page 5. In the revision manuscript we provided the detailed explanation for complexity analysis of our algorithms. 
	

	\item Regarding accuracy: in addition to theoretical bounds that we provided for our algorithms (optimality of \fcs, approximation ratio for \ics when all EVs have same arrival time, and competitive ratio of \focs), in simulation section we compared each algorithm to its optimal solution with a confidence level of $95\%$ to assure the accuracy of results. The results show that our online algorithms (i.e., \iocs and \focs) are, on average, around $80\%$ of their optimal solution while the offline algorithm \ics is always more than $90\%$ of the optimal in different scenarios. For more details, please refer to Section VII of the manuscript on Pages~9-12.
\end{itemize}

\vspace{5mm}
{\color{blue}
\noindent\textbf{Comment 3.4}\\
Overall, the topic is worth investigation. But, to the best of the reviewer's knowledge, this paper presents an artificial concept that couples microgrid revenue maximization with EV charging. It is quite complicated technically, but that cannot hide the fact that fundamental real issues that any real-world microgrid operator faces have not been properly addressed.}

\vspace{7mm}
\noindent\textbf{Response:}\\
Thanks for the comment.
Among several issues that the future microgrids are facing, we believe that their policies in managing huge EV charging demand are critical. The reason is that there is a rapid growth in both EV usage and EV battery size (See Section I-B of the revised manuscript for more details and real-world numbers). Considering this general scenario, the aim of our paper is to solve EV scheduling problem in a small microgrid composing of multiple charging stations owned by a single utility provider such as in a university or building. 
%It is assumed in the paper that the microgrid receives a certain amount of electricity from main grid in a hourly manner and each charging station also has its own constraints on the amount of electricity it can deliver to EVs. We also assume that a central controller is responsible to optimize scheduling of EVs in all charging stations. We did not insert some parameters into our formulation as they have no effect on the final solution such as location of the charging stations. Overall, we agree that the problem formulation is  standard, but we believe that this does not affect the practicality of the solutions.

\vspace{5mm}
{\color{blue}
\noindent\textbf{Comment 3.5}\\
The quality of this paper can be significantly improved. Overall, the reviewer does NOT recommend the acceptance of this paper as is. Another concern is that IEEE TVT may not be a suitable venue for this paper.}

\vspace{7mm}
\noindent\textbf{Response:}\\
To address the reviewer's concern, in addition to the new technical improvements, we carefully improved the organization and presentation of the paper. In summary, we believe that the revised manuscript is significantly improved in terms of technical contribution (by proposing more practical online algorithms, and further complexity analysis of the previous algorithms) and providing more related works and distinguishing our work among the previous work by highlighting unique and important challenges that we are tackling. 

%%%%%%%%%%%%%%%%
\newpage
\section{Reviewer $\# 4$}
{\color{blue}This paper proposes fractional and integral revenue models for EV charging scheduling problems in multiple parking stations. The paper is well organized and the technical idea is clearly presented with good integrity. The authors gave the proof of optimum of the proposed algorithm for fractional revenue model, and analyzed the complexity of their algorithms. The reviewer recommends this paper after the revision based on the following suggestions.}

\subsection{Major Comments}

{\color{blue}
\noindent\textbf{Comment 4.1}\\
The authors may elaborate the pricing policy applied by the parking station or utility which governs these parking stations.}

\vspace{2mm}
\noindent\textbf{Response:}\\
Thanks for pointing this precious comment. The variable $v_i$ defined in the paper is valuation of EV $i$ when it receives its demand $D_i$. The valuation $v_i$ in simple and default case is the price imposed by charging station that the owner of EV $i$ should pay to receive its demand, $D_i$ (as it is in the paper). In this case, the charging station is allowed to use any pricing policy as it does not affect our algorithms. With this interpretation, our algorithms solve revenue maximization problem for utility provider. $v_i$ can also represent the valuation of demand from the user's point of view indicating that how much the user will be happy if it receives its submitted demand. With this interpretation, our algorithms try to maximize users' satisfaction or users' social welfare. Therefore, our proposed algorithms are general and work for any valuation setting that is proposed by either EVs or charging stations.

\vspace{5mm}
{\color{blue}
\noindent\textbf{Comment 4.2}\\
This paper only discusses the charging scheduling problem in a static scenario. Existing work such as “QoE-aware Power Management in Vehicle-to-Grid Networks: a Matching-theoretic Approach” in IEEE Transactions on Smart Grid has discussed the EV charging scheduling problem under the mobile scenario. These existing work should be addressed in Section II.}

\vspace{2mm}
\noindent\textbf{Response:}\\
We thank the reviewer for suggesting this related study. In the revised version, we have considered dynamic (online) scenario, as well (for further explanations, please read response to Comment 1.1). In addition, we have added subsection II-4 to review online EV charging algorithms.
The suggested work is related to our studied problem and focuses on assignment of EVs to different charging stations in a network of charging stations (not a real microgrid) assuming that the travel information of EVs is known. We added this work to our references and cited it in Section I and Section II. 

%{\color{red}: I think we can mention that in the revised version in addition to static scenario, we have considred dynamic (online) scenario and refer to the comments above for further explanation...}

\vspace{2mm}
{\color{blue}
\noindent\textbf{Comment 4.3}\\
The authors should give the definition of the valuation of EVs in detail in the paper. Does the valuation mean the price that an EV will be charged after the charging is completed?}

\vspace{7mm}
\noindent\textbf{Response:}\\
We believe that the answer to this comment is given in response of Comment 4.1.


%%%%%%%%%%%%
\newpage
\section{Reviewer $\# 5$}
{\color{blue}The paper is focusing on revenue maximization of EV parking stations which are part of a microgrid having an overall capacity limit. The paper is written with a very clear language and a very good organization.

The paper formulated two optimization problems named fractional revenue model and integral revenue model, both of which have a high amount of applicability in a realistic parking station environment. The stated optimization problems are sound and very easy to understand. Next, the paper proposes one algorithm for each of the two stated optimization problems the ISCS and FSCS. Both algorithms are explained in detail. Then, the performance of these two proposed algorithms have been evaluated and  compared withe the optimal result for the integral revenue problem and a greedy algorithm from the literature via simulations. The results show that, the proposed algorithms work quite nice in total revenue, actual peak, and \% of fully charged EVs.

Overall, this is a very good paper focusing on a timely topic.}

\subsection{Major Comments}

\vspace{2mm}
{\color{blue}
\noindent\textbf{Comment 5.1}\\
On page 8, it is stated that the slackness parameter is selected to be as 1.2. Then, in section VI.D, the authors show the results of selecting this slackness and elaborate on its effects. It would be really good, if on page 8 it is stated that the effect of slackness will be investigated in section VI.D. Otherwise, the reader mistakenly thinks that slackness is selected arbitrarily as 1.2 which confuses the reader.}

\vspace{7mm}
\noindent\textbf{Response:}\\
We thank the reviewer for pointing this out. We added the required explanation on Page $10$ of the revised version 
%{\color{red} check the page numbers in the final round, this is not correct.}

\vspace{5mm}
{\color{blue}
\noindent\textbf{Comment 5.2}\\
As explained in section VI.A, the paper uses a probabilistic model for the arrival and departure times, the required recharge amount for each EV. There are some publications in the EV recharging literature which focuses on using realistic traces as an input instead of a probabilistic one. The usage of a realistic vehicular mobility trace has been briefly mentioned in the conclusion paper. It might be worthwhile to elaborate more on this topic and explain what the authors expect if such a realistic trace would be used.}

\vspace{5mm}
\noindent\textbf{Response:}\\
Thanks for highlighting this point. We agree that using realistic vehicular mobility trace can better expose the strengths and weaknesses of the proposed methods. In light of this comment, we developed and added two online algorithms for both integral and fractional revenue models to the revised manuscript where the details can be found in Section VI. Also, we re-run all simulation scenarios in Section VII to include the new algorithms. The results reveal that the performance of the online solutions are around $80\%$ of the optimal offline solution. The Abstract and Conclusion sections are also updated accordingly.


%We agree with the reviewer 

%Thanks for this useful comment.  

%We thank the reviewer for pointing this out. 

%Thanks for suggesting this point. 

%We acknowledge the reviewer for suggesting this point. We agree that the 


%We sincerely appreciate you for taking the time to review our paper. Following the concerns and suggestions, the paper has carefully been revised to address the reviewer's comments properly. We also tried to briefly reflect all the clarifications in the revised 4-page manuscript. 

%Thanks. We agree that .....

%We agree that ....

%We thank the reviewer for pointing this out. 

%Thanks for highlighting this point. 

%We thank the reviewer for pointing this precious comment. 

%Thanks for pointing this issue.  

%We thank the reviewer for suggesting this point. 


%\begin{itemize}
%\item[[A7]] D. G. Luenberger, Optimization by Vector Space Methods, John Wiley and Sons, Inc., New York, 1969.   
%\end{itemize}

%\vspace{5mm}
%\noindent\textbf{Response:}\\
%Thanks. We believe that the response to this comment is given within our response to Comment 3.11 above. 
\begin{comment}
			\bibliographystyle{IEEEtranS}
			
			%	\bibliography{IEEEabrv,references}
			\begin{thebibliography}{1}
								
\bibitem{Malhotra}
A. Malhotra, G. Binetti, A. Davoudi and I. D. Schizas, ``Distributed Power Profile Tracking for Heterogeneous Charging of Electric Vehicles,'' \emph{IEEE Trans. Smart Grid}, no. 99, pp. 1-10, 2016.

\bibitem{Jiang}
				B. Jiang, and Y. Fei, ``Decentralized Scheduling of PEV On-Street
				Parking and Charging for Smart Grid Reactive Power Compensation,'' in \emph{Proc.
					IEEE PES Innov. Smart Grid Technol. (ISGT)}, Washington, DC, USA, Feb. 2013.
					
\bibitem{Wen}
				C-K. Wen, J-C. Chen, J-H. Teng, and P. Ting, ``Decentralized Plug-in Electric Vehicle Charging Selection Algorithm in Power Systems,'' \emph{IEEE Trans. Smart Grid}, vol. 3, no. 4, pp. 1779-1789, 2012.
				
\bibitem{YCao}   
				Y. Cao, and N. Wang, ``Towards Efficient Electric Vehicle Charging Using VANET-Based Information Dissemination,'' \emph{IEEE Trans. on Vehicular Technology}, no. 99   pp. 1-1, 2016.
				%In this article, we aim to improve drivers comfort, e.g., to minimize EV’s charging waiting time. Particularly, the CS-selection decision on where to charge is made by individual EV for privacy and scalability benefits.
				
\bibitem{TZhang}   
				T. Zhang, W. Chen, and Z. Han, ``Charging Scheduling of Electric Vehicles With Local Renewable Energy Under Uncertain Electric Vehicle Arrival and Grid Power Price,'' \emph{IEEE Trans. on Vehicular Technology}, vol. 63, no. 6, pp. 2600-2612, 2014.
				%In this paper, we consider delay-optimal charging scheduling of the electric vehicles (EVs) at a charging station with multiple charge points. The goal is to minimize the mean waiting time (in ueue) for EVs under the long-term constraint on the cost.   
				
				
\bibitem{Akhavan}
				E. Akhavan-Rezai, M.F. Shaaban, E.F. El-Saadany, and F. Karray ``Online Intelligent Demand Management of Plug-In Electric Vehicles in Future Smart Parking Lots,'' \emph{IEEE Systems Journal}, no. 99, pp. 1-12, 2015.
\bibitem{WTang}
W. Tang, and Y. J. Zhang,  ``A Model Predictive Control Approach for Low-Complexity Electric Vehicle Charging Scheduling: Optimality and Scalability,'' \emph{IEEE Trans. on Power Systems}, 2017.

\bibitem{yds}
F. F. Yao, A. J. Demers, and S. Shenker, ``A scheduling model for reduced CPU energy,'' in \emph{ Proc. of 36th IEEE Symposium on Foundations of Computer Science}, 1995.

\bibitem{Lee}
				W. Lee, L. Xiang, R. Schober, and V. W. S. Wong, ``Electric Vehicle Charging Stations With Renewable Power Generators: A Game Theoretical Analysis,'' \emph{IEEE Trans. Smart Grid}, vol. 6, no. 22, pp. 608-617 , 2015.
	

\bibitem{Zhao}
				S. Zhao, X. Lin, and M. Chen, ``Peak-Minimizing Online EV Charging: Price-of-Uncertainty and Algorithm Robustification,'' in \emph{Proc. of IEEE INFOCOM} ,2015.
				
\bibitem{Karfopoulos}
				E. L. Karfopoulos, and N. Hatziargyriou, ``Distributed Coordination of Electric Vehicles Providing V2G Services,'' \emph{IEEE Trans. on Power Systems}, vol. 31, no. 1, pp. 329-338, 2015.

\bibitem{Chen}
				S. Chen and L. Tong, ``iEMS for Large Sscale Charging of Electric Vehicles
				Architecture and Optimal Online Scheduling,'' \emph{in Proc. IEEE Int. Conf.
					Smart Grid Commun. (SmartGridComm)}, 2012.
								
				%%	
				
\bibitem{Zen}
M. Zen, S. Leng, Y. Zhang, and J. He,``QoE-aware Power Management in Vehicle-to-Grid Networks: a Matching-theoretic Approach,'' \emph{IEEE Trans. of Smart Grids,} vol. PP, no. 99, 2016.					

			
			\end{thebibliography}
\end{comment}

\end{document}



6741 460 573 918
